\documentclass{article}
\usepackage[utf8]{inputenc}
\usepackage{xcolor}
\usepackage{listings}
\usepackage{graphicx}
\usepackage{etoolbox}
\usepackage{titlesec}
\usepackage{multirow}
\usepackage{booktabs}
\usepackage{fancyvrb}
\usepackage{amsmath}
\usepackage{amssymb}
\usepackage[spanish]{babel}
\usepackage[most]{tcolorbox}
\usepackage[hidelinks]{hyperref}
\usepackage[nottoc,numbib]{tocbibind}
\hyphenchar\font=-1

\hypersetup{
    allcolors=black
}

\renewcommand{\contentsname}{Índice}
\renewcommand\refname{Referencias}

% set the default code style
\lstset{
    language = C,
    frame=tblr, % draw a frame at the top and bottom of the code block
    rulecolor=\color{black},
    tabsize=4, % tab space width
    breakatwhitespace=true,         
    breaklines=true,                 
    captionpos=b,                    
    keepspaces=false,
    showspaces=false,                
    showstringspaces=false,
    showtabs=false,
    basicstyle=\small\ttfamily\bfseries,
    commentstyle=\color{red}, % comment color
    keywordstyle=\color{blue}, % keyword color
    stringstyle=\color{orange} % string color
}
\title{Trabajo Práctico 4 - TBD}
\date{16-04-2018}
\author{
    Román Castellarín\\
    \and
    Gianfranco Paoloni\\
    \and
    Juan Ignacio Suarez
}


\newtcblisting{commandshell}{colback=black,colupper=white,colframe=black,
listing only,listing options={language=sh,  keywordstyle=\color{white}, basicstyle=\small\ttfamily\bfseries, breaklines=true, keepspaces=true, breakatwhitespace=false,  captionpos=b, showstringspaces=false, showtabs=false }}

\begin{document}

\begin{titlepage}
    \centering
    \includegraphics[width=0.4\textwidth]{unrlogo}\par\vspace{1cm}
    {\scshape\LARGE Teoría de Bases de Datos \par}
    {\scshape\Large Trabajo práctico 4\par}
    \vspace{1.5cm}
    {\huge\bfseries Normalización \par}
    \vspace{2cm}
    {\Large\itshape  Román Castellarín \\
                     Gianfranco Paoloni \\
                     Juan Ignacio Suarez \par}
    \vfill

% Bottom of the page
    {\large \today \par}
\end{titlepage}

\tableofcontents

\newpage
\section{Introducción}
En este documento se describe la definición e implementación de varios algoritmos sobre conjuntos de atributos y de dependencias funcionales. Para realizar el testeo de estos algoritmos se proponen 5 sets de prueba:

\begin{Verbatim}[frame=single]
    Set 1
    R = {A, B, C, D}
    F = {A->B, CB->A, B->AD}
\end{Verbatim}

\begin{Verbatim}[frame=single]
    Set 2
    R = {A, B, C, D, E, F}
    F = {AB->C, BD->EF}
\end{Verbatim}

\begin{Verbatim}[frame=single]
    Set 3
    R = {A, B, C, D, E, F, G, H, I, J}
    F = {AB->C, BD->EF, AD->GH, A->I, H->J}
    A = {B, D}
\end{Verbatim}

\begin{Verbatim}[frame=single]
    Set 4
    R = {A, B, C, D, E, F, G, H}
    F = {A->BC, C->D, D->G, H->E, E->A, E->H}
    A = {A, C}
\end{Verbatim}

\begin{Verbatim}[frame=single]
    Set 5
    R = {A, B, C, D, E, F, G}
    F = {A->G, A->F, B->E, C->D, E->A, D->B, GF->C}
    A = {F, G}
\end{Verbatim}

Cada uno de estos sets se proveen bajo la sintaxis particular de nuestro programa en el directorio \verb|Examples|, en los archivos \verb|set1.txt|, \verb|set2.txt|, \verb|set3.txt|, \verb|set4.txt| y \verb|set5.txt|.

\subsection{Estructura del programa}
El programa ha sido implementado en \verb|C++11|, incluye seis archivos principales: \verb|main.cpp|, \verb|funcdependency.h|, \verb|funcdependency.cpp|, \verb|set_closure.cpp|, \verb|attr_closure.cpp| y \verb|candidate_keys.cpp|.

\begin{itemize}
    \item \verb|main.cpp|: mini-parser de archivos de entrada y llamadas correspondientes
    a las funciones requeridas en el trabajo práctico.
    \item \verb|funcdependency.h|: definiciones de conjuntos de atributos y dependencias, así también como operadores y funciones requeridas para la implementación general.
    \item \verb|funcdependency.cpp|: implementación de las definiciones generales previamente mencionadas.
    \item \verb|set_closure.cpp|: implementación para encontrar la clausura de un conjunto de dependencias funcionales.
    \item \verb|attr_closure.cpp|: similar al anterior aplicado a la clausura de atributos.
    \item \verb|candidate_keys.cpp|: implementación para determinar todas las posibles claves candidatas de un conjunto de atributos y las dependencias funcionales correspondientes al conjunto.
\end{itemize}

\subsection{Compilación}
Para compilar el programa basta con compilar con \texttt{g++} (versión 2011 o más moderna) todos los archivos mencionados en la sección anterior. Dado que en la implementación probamos casos sobre muchos subconjuntos es ideal compilar con optimización de tipo 3 (-O3). Además, crearemos un ejecutable con el nombre \verb|fd| para poder tener una consistencia con los ejemplos siguientes. Todo esto se puede hacer en una línea con el comando

\begin{commandshell}
> g++ -std=c++11 -O3 *.cpp -o fd
\end{commandshell}

Si tuvimos éxito debemos tener un archivo ejecutable llamado fd en el mismo directorio.

\subsection{Ejecución}
Suponiendo que el ejecutable compilado se llama \verb|fd|, la ejecución es tan simple como correr

\begin{commandshell}
> ./fd archivo_de_entrada
\end{commandshell}
\small{será fd.exe en vez de ./fd si estamos en Windows} \\

Dado que calcular el conjunto de cierre de dependencias funcionales resulta una operación muy costosa como veremos luego, hemos incluído un comando opcional \verb|-countFD| si se desea contar la cardinalidad de dicho conjunto. La ejecución con esta opción sería

\begin{commandshell}
> ./fd archivo_de_entrada -countFD
\end{commandshell}

\newpage
\section{Cierre de conjuntos de dependencias funcionales $F^{+}$}
\subsection{Algoritmo}
Sea $F$ un conjunto de dependencias funcionales definido sobre $R$. El cierre de $F$, denotado por $F^{+}$, es el conjunto de todas las dependencias funcionales que $F$ implica lógicamente. El siguiente algoritmo escrito en pseudocódigo basado en los axiomas de Armstrong nos permite calcular $F^{+}$:

\begin{Verbatim}[frame=single]
    resultado := F;
    
    for each alfa in P(R) do
        aplicar las reglas de reflexividad a alfa;
        añadir las nuevas DF obtenidas a resultado;
        
    while (hay cambios en resultado) do
        for each DF f in resultado do
            aplicar las reglas de aumentatividad a f;
            añadir las nuevas DF obtenidas a resultado;
            
    for each DF f1 in resultado do
        for each DF f2 in resultado do
            if f1 y f2 pueden combinarse por transitividad then
                añadir la nueva DF a resultado;
                
    return resultado;
\end{Verbatim}

\subsection{Implementación}
La implementación de este algoritmo se puede encontrar en el archivo \verb|set_closure.cpp|

\newpage
\section{Cierre de conjuntos de atributos $\alpha^{+}$}
\subsection{Algoritmo}
Sea $\alpha$ un conjunto de atributos. Al conjunto de todos los atributos determinados funcionalmente por $\alpha$ bajo un conjunto $F$ de dependencias funcionales se lo denomina cierre de $\alpha$ bajo $F$ y se denota mediante $\alpha^{+}$. A continuación se muestra un algoritmo escrito en pseudocódigo para calcular $\alpha^{+}$ a partir de $F$ y $\alpha$:

\begin{Verbatim}[frame=single]
    resultado := alfa;
    
    while (hay cambios en resultado) do
        for each dependencia funcional b->c in F do
            if b está contenido en resultado then
                resultado := resultado UNION c;
    
    return resultado;
\end{Verbatim}

\subsection{Implementación}
La implementación de este algoritmo se puede encontrar en el archivo \verb|attr_closure.cpp|


\newpage
\section{Conjunto de claves candidatas $S_{K}$}
\subsection{Algoritmo}
Dada una relación $R$ y un conjunto de dependencias funcionales $F$ queremos encontrar el conjunto de todas sus claves candidatas $S_{K}$. Recordemos la definición de clave candidata:

Una clave candidata $K$ de una relación $R$ es una superclave minimal para dicha relación, es decir, es un conjunto de atributos $K$ que cumple:
\begin{enumerate}
    \item $K$ es superclave, es decir que $K^{+} = R$
    \item $K$ es minimal, es decir que  $\nexists Y \subset K\ |\ Y^{+} = R$. Notar que la inclusión es estricta, pues $Y$ es un subconjunto propio de $K$.
\end{enumerate}

Dicho esto podemos entonces escribir un algoritmo (en pseudocódigo) para calcular $S_{K}$ a partir de $R$ y $F$:

\begin{Verbatim}[frame=single]
    resultado := vacío;
    
    for each X in P(R) do
        if X+ == R then
            esCandidato := true;
            
            for each Y in P(X) - X do
                if Y+ == R then
                    esCandidato := false;
            
            if esCandidato == true then
                añadir X a resultado
                
    return resultado;
\end{Verbatim}

\subsection{Implementación}
La implementación de este algoritmo se puede encontrar en el archivo \verb|candidate_keys.cpp|


\newpage
\section{Salidas del programa en los sets de prueba}
Para verificar la correcta ejecución de los algoritmos, a continuación se presenta la ejecución del programa en cada uno de los sets de prueba descritos en la introducción.

\subsection{Set 1}
Corriendo el programa sobre el set 1 se obtiene

\begin{commandshell}
> ./fd Examples/set1.txt -countFD
A+ closure: empty
F+ size: 137
Candidate keys:
AC
BC
\end{commandshell}

Por lo que la cardinalidad del conjunto $F^{+}$ es 137, y el conjunto de claves candidatas $S_{K}$ resulta $\{AC,BC\}$.


\subsection{Set 2}
Corriendo el programa sobre el set 2 se obtiene

\begin{commandshell}
> ./fd Examples/set2.txt -countFD
A+ closure: empty
F+ size: 1081
Candidate keys:
ABD
\end{commandshell}

Por lo que la cardinalidad del conjunto $F^{+}$ es 1081, y el conjunto de claves candidatas $S_{K}$ resulta $\{ABD\}$.


\subsection{Set 3}
Corriendo el programa sobre el set 3 se obtiene

\begin{commandshell}
> ./fd Examples/set3.txt
A+ closure: BDEF
Candidate keys:
ABD
\end{commandshell}

Por lo que la clausura de atributos del conjunto $A$ resulta $A^{+} = \{BDEF\}$, y el conjunto de claves candidatas $S_{K}$ resulta $\{ABD\}$


\subsection{Set 4}
Corriendo el programa sobre el set 4 se obtiene

\begin{commandshell}
> ./fd Examples/set4.txt
A+ closure: ABCDG
Candidate keys:
EF
FH
\end{commandshell}

Por lo que la clausura de atributos del conjunto $A$ resulta $A^{+} = \{ABCDG\}$, y el conjunto de claves candidatas $S_{K}$ resulta $\{EF,FH\}$


\subsection{Set 5}
Corriendo el programa sobre el set 5 se obtiene

\begin{commandshell}
> ./fd Examples/set5.txt
A+ closure: ABCDEFG
Candidate keys:
A
B
C
D
E
FG
\end{commandshell}

Por lo que la clausura de atributos del conjunto $A$ resulta $A^{+} = \{ABCDEFG\}$, y el conjunto de claves candidatas $S_{K}$ resulta $\{A,B,C,D,E,FG\}$


\nocite{*}
\bibliography{referencias}
\bibliographystyle{plain}

\end{document}